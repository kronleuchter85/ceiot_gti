\documentclass[
11pt, % The default document font size, options: 10pt, 11pt, 12pt
%codirector, % Uncomment to add a codirector to the title page
]{charter}




% El títulos de la memoria, se usa en la carátula y se puede usar el cualquier lugar del documento con el comando \ttitle
\titulo{Solución IoT para robot de exploración ambiental de datos críticos con almacenamiento en blockchain}

% Nombre del posgrado, se usa en la carátula y se puede usar el cualquier lugar del documento con el comando \degreename
\posgrado{Maestría en Internet de las Cosas}
%\posgrado{Carrera de Especialización en Internet de las Cosas}
%\posgrado{Carrera de Especialización en Intelegencia Artificial}
%\posgrado{Maestría en Sistemas Embebidos}
%\posgrado{Maestría en Internet de las cosas}

% Tu nombre, se puede usar el cualquier lugar del documento con el comando \authorname
\autor{Ing. Gonzalo F. Carreño}

% El nombre del director y co-director, se puede usar el cualquier lugar del documento con el comando \supname y \cosupname y \pertesupname y \pertecosupname
\director{Esp. Ing. Sergio Alberino}
\pertenenciaDirector{UTN - FRBA}
% FIXME:NO IMPLEMENTADO EL CODIRECTOR ni su pertenencia
%\codirector{CODIRECTOR} % para que aparezca en la portada se debe descomentar la opción codirector en el documentclass
%\pertenenciaCoDirector{FIUBA}

% Nombre del cliente, quien va a aprobar los resultados del proyecto, se puede usar con el comando \clientename y \empclientename
\cliente{Esp. Lic. Mariano Landini}
\empresaCliente{UBA - FCE}

% Nombre y pertenencia de los jurados, se pueden usar el cualquier lugar del documento con el comando \jurunoname, \jurdosname y \jurtresname y \perteunoname, \pertedosname y \pertetresname.
\juradoUno{Nombre y Apellido (1)}
\pertenenciaJurUno{pertenencia (1)}
\juradoDos{Nombre y Apellido (2)}
\pertenenciaJurDos{pertenencia (2)}
\juradoTres{Nombre y Apellido (3)}
\pertenenciaJurTres{pertenencia (3)}
\fechaINICIO{22 de junio de 2024}		%Fecha de inicio de la cursada de GdP \fechaInicioName
\fechaFINALPlan{17 de agosto de 2024} 	%Fecha de final de cursada de GdP
\fechaFINALTrabajo{20 de octubre de 2025}	%Fecha de defensa pública del trabajo final


\begin{document}

\maketitle
\thispagestyle{empty}
\pagebreak


\thispagestyle{empty}
{\setlength{\parskip}{0pt}
\tableofcontents{}
}
\pagebreak


\section*{Registros de cambios}
\label{sec:registro}


\begin{table}[ht]
\label{tab:registro}
\centering
\begin{tabularx}{\linewidth}{@{}|c|X|c|@{}}
\hline
\rowcolor[HTML]{C0C0C0}
Revisión & \multicolumn{1}{c|}{\cellcolor[HTML]{C0C0C0}Detalles de los cambios realizados} & Fecha      \\ \hline
0      & Creación del documento.                                 								&\fechaInicioName \\ \hline
1      & Se aplican correcciones y completan capítulos restantes								& 9 de agosto de 2024 \\ \hline
2      & Se aplican correcciones y sugerencias								& 16 de agosto de 2024 \\ \hline
\end{tabularx}
\end{table}

\pagebreak



\section*{Acta de constitución del proyecto}
\label{sec:acta}

\begin{flushright}
Buenos Aires, \fechaInicioName
\end{flushright}

\vspace{2cm}

Por medio de la presente se acuerda con el \authorname\hspace{1px} que su Trabajo Final de la \degreename\hspace{1px} se titulará ``\ttitle'', consistirá en \textcolor{black}{la implementación de un sistema embebido de robot de exploración ambiental conectado a un sistema back-end en la nube pública capaz de persistir los datos en una red blockchain}, y tendrá un presupuesto preliminar estimado de \textcolor{black}{\$ 3.720,00} dólares estadounidenses (equivalente a \$ 3.348.000,00 pesos argentinos tomando la cotización del Banco de la Nación Argentina de \$ 900 por dólar vigente al 16 de agosto de 2024) y \textcolor{black}{615} horas de trabajo, con fecha de inicio \fechaInicioName\hspace{1px} y fecha de presentación pública a definir entre octubre y diciembre de 2025.

Se adjunta a esta acta la planificación inicial.

\vfill

% Esta parte se construye sola con la información que hayan cargado en el preámbulo del documento y no debe modificarla
\begin{table}[ht]
\centering
\begin{tabular}{ccc}
\begin{tabular}[c]{@{}c@{}}Dr. Ing. Ariel Lutenberg \\ Director posgrado FIUBA\end{tabular} & \hspace{2cm} & \begin{tabular}[c]{@{}c@{}}\clientename \\ \empclientename \end{tabular} \vspace{2.5cm} \\
\multicolumn{3}{c}{\begin{tabular}[c]{@{}c@{}} \supname \\ Director del Trabajo Final\end{tabular}} \vspace{2.5cm} \\
%\begin{tabular}[c]{@{}c@{}}\jurunoname \\ Jurado del Trabajo Final\end{tabular}     &  & \begin{tabular}[c]{@{}c@{}}\jurdosname\\ Jurado del Trabajo Final\end{tabular}  \vspace{2.5cm}  \\
%\multicolumn{3}{c}{\begin{tabular}[c]{@{}c@{}} \jurtresname\\ Jurado del Trabajo Final\end{tabular}} \vspace{.5cm}                                                                   
\end{tabular}
\end{table}




\section{1. Descripción técnica-conceptual del proyecto a realizar}
\label{sec:descripcion}

El presente proyecto es un emprendimiento personal que busca integrar el dispositivo robótico de exploración ambiental controlable a distancia desarrollado en el marco de la Carrera de Especialización en Sistemas Embebidos con un backend para el análisis y explotación de datos en la nube pública y una red blockchain para el almacenamiento inmutable de datos críticos.

\textbf{Estado del arte:}
los robots exploradores son los dispositivos robotizados que han sido creados con el fin de reconocer y explorar un lugar o terreno siendo capaces de moverse de forma autónoma o controlados por personas a control remoto. Su objetivo es evitar poner en riesgo la vida de los humanos, ya sea debido a que el lugar es inaccesible o porque se encuentra en una zona contaminada.
Algunos de los tipos de robots exploradores más conocidos son los espaciales, de minas, de rescate en catástrofes, de tuberías, acuáticos y/o submarinos y de suelos.

Estos dispositivos, forman parte de una solución arquitectónica IoT más amplia, cumpliendo el rol de nodos “edge”.  Una vez implementados y funcionando envían lecturas del reconocimiento y operaciones realizadas a sistemas de procesamiento y almacenamiento de datos que conforman el backend, generalmente en la nube, desde donde se puede visualizar los resultados obtenidos, sus métricas derivadas, realizar trazabilidad de las operaciones así como almacenar y gobernar los datos generados.

En situaciones en las que el robot es utilizado para explorar y monitorear un área ambientalmente sensible, como una reserva natural o un sitio afectado por un desastre ecológico y su misión es recopilar datos críticos tales como niveles de contaminación, temperatura, humedad, y calidad del aire. Es de gran importancia almacenar estas mediciones de una manera en la que se pueda asegurar la integridad y transparencia de los datos, como por ejemplo, en una cadena de bloques (blockchain).

Una arquitectura blockchain se basa en el procesamiento y almacenamiento de transacciones agrupadas en bloques encadenados e inmutables de forma distribuida entre los nodos de una red en lo que se conoce como un \textit{distributed ledger}. De esta manera se puede asegurar la integridad de los datos ya que los registros generados no se pueden modificar una vez creados. Además, como la red es accesible entre los actores intervinientes en el caso de uso, ya sea pública o privada y con o sin permisos, se puede garantizar la transparencia de los datos.
La mayoría de las redes blockchain constan de la tecnología para la implementación de código backend ejecutable en la red, que aunque su nombre puede cambiar dependiendo de la red en la cual se implementan, usualmente se los conoce como \textit{Smart Contracts}. La ejecución de estos componentes es realizada por los nodos de la red en el proceso que se conoce como minería, y como tal es una actividad que requiere el pago de un \textit{fee} (comisión) conocido como \textit{gas} medido en diferentes unidades dependiendo de la red y normalmente pagable desde una cuenta nominada en el token de la red asociada a la aplicación o propietario de los \textit{smart contracts}.
La  forma de interactuar con los \textit{smart contracts} en un caso de uso interactivo desde afuera de la red, se realiza a través de otro componente conocido como dApps \textit{(de-centralized applications)} que haciendo uso de ciertas bibliotecas de Web3 invocan a estos para almacenar y obtener datos en y desde el \textit{ledger}.

\vspace{8em}

En la figura \ref{fig:IotArquitecturaGlobal} se puede apreciar una posible arquitectura del sistema final.

\begin{center}
 \includegraphics[scale=0.25]{Figuras/IoTProject-Page-1.drawio}
 \captionof{figure}{Arquitectura del sistema}
 \label{fig:IotArquitecturaGlobal}
\end{center}


El hardware utilizado para la solución de IoT propuesta es un robot de exploración ambiental de control inalámbrico, desarrollado en la Carrera de Especialización en Sistemas Embebidos de la Universidad de Buenos Aires. En la figura \ref{fig:RobotHardware} puede apreciarse una fotografía del mismo.

\begin{center}
 \includegraphics[scale=0.5]{Figuras/Robot_y_Joystick_1}
 \captionof{figure}{Robot de exploración ambiental.}
 \label{fig:RobotHardware}
\end{center}




\vspace{25px}



\section{2. Identificación y análisis de los interesados}
\label{sec:interesados}
A continuación, se enumeran los diferentes roles e individuos que participarán en el proyecto.
\begin{table}[ht]
%\caption{Identificación de los interesados}
%\label{tab:interesados}
\begin{tabularx}{\linewidth}{@{}|l|l|X|l|@{}}
\hline
\rowcolor[HTML]{C0C0C0}
Rol           	& Nombre y Apellido & Organización 	& Puesto 	\\ \hline
Cliente       	& \clientename      &\empclientename	&  Usuario final      	\\ \hline
Responsable   	& \authorname       & UTN - FRBA        	& Alumno 	\\ \hline
Orientador    	& \supname	      	& \pertesupname 	& Director Trabajo final \\ \hline

\end{tabularx}
\end{table}


\section{3. Propósito del proyecto}
\label{sec:proposito}

El propósito de este proyecto es desarrollar una solución IoT para un robot de exploración ambiental para casos de uso de datos críticos en los que sea necesario contar con almacenamiento inmutable y transparente para todas las partes en una arquitectura blockchain.
Por otra parte, se pretende volcar en un desarrollo concreto y de aplicación industrial los conocimientos adquiridos durante la cursada de la Maestría en Internet de las Cosas.

\section{4. Alcance del proyecto}
\label{sec:alcance}


Contando con el hardware mencionado anteriormente que auspicia de componente edge en una arquitectura IoT, se propone la implementación de la plataforma tecnológica que alcanza:

\begin{itemize}
	\item La publicación del endpoint MQTT para la recepción de los datos enviados por el robot.
	\item La adaptación del sistema embebido del robot de exploración ambiental para la conexión segura con el backend vía MQTT.
	\item La arquitectura e implementación de los sistemas backend y el modelo de datos necesario para el almacenamiento de las mediciones enviadas por el robot.
	\item La arquitectura, implementación y despliegue de la dApp y \textit{smart contracts} necesarios para el almacenamiento de las mediciones en una red Blockchain (a definir).
	\item La definición de métricas agregadas de valor y posterior arquitectura e implementación de los sistemas analíticos para procesar de forma \textit{batch} y/o \textit{real-time} (dependiendo de las métricas a definir) utilizando herramientas de procesamiento paralelo basadas en big data.
	\item La implementación de la interfaz gráfica para poder visualizar los datos enviados y analíticas calculadas.

\end{itemize}



\section{5. Supuestos del proyecto}
\label{sec:supuestos}

Para el desarrollo del presente proyecto se supone que:

\begin{itemize}
	\item Será posible tener acceso a redes de desarrollo y testing de forma gratuita mediante la obtención \textit{tokens} de prueba por medio de \textit{Faucets}.
	\item Será posible experimentar, desarrollar y hacer testing del backend cloud con el presupuesto estimado.
	\item Será posible contar con la capa gratuita en el servicio cloud utilizado.
	\item Arquitectónicamente resultará viable implementar la solución propuesta.
	\item Se dispondrá del conjunto de bibliotecas, drivers y APIs de bajo nivel para el desarrollo de las funcionalidades planteadas en el alcance sin ser necesario el desarrollo de drivers y dichos componentes de bajo nivel. Además, tanto estos componentes de software como los  \textit{open source} de la comunidad de software libre utilizados durante el desarrollo del producto, se encontrará estable para que su integración en el proyecto no resulte en desvíos.	
	\item Tanto el prototipado de los componentes de software del sistema embebido como el ensamblado de los componentes de hardware del dispositivo no producirán desvíos considerables en el plan.
	\item No habrá desvíos no contemplados en el plan que impidan o demoren entregas en el proyecto.
	\item El comité académico encargado de la corrección tendrá disponibilidad para realizar la evaluación en las fechas planificadas de entrega.
	\item El director asignado tendrá la disponibilidad de tiempo para darle seguimiento al proyecto.
	\item El alumno contará con una disponibilidad de entre 3 y 5 horas diarias (incluyendo fines de semana) para el desarrollo del proyecto en el tiempo convenido.
	\item Los materiales, componentes, software de terceros y servicios cloud utilizados funcionaran de forma óptima y de acuerdo a lo esperado.
	\item El robot desarrollado seguirá funcionando de forma estable sin ser necesario su ajuste, reparacion o modificacion a nivel hardware o sistema base (fuera de lo planificado para la integración con MQTT).
	\item Los recursos no directamente relacionados con el desarrollo del proyecto, pero utilizados en este, funcionarán adecuadamente y en caso de falta de suministro (por ejemplo el servicio de Internet) o avería (por ejemplo, en el caso de la computadora utilizada) la resolución será expeditiva no suponiendo un desvío en el plan.
	\item No sucederán nuevos eventos de impacto global (pandemia, guerras, etc) durante el desarrollo del proyecto que impliquen una demora o imposibilidad en la entrega. 
\end{itemize}



\section{6. Requerimientos}
\label{sec:requerimientos}

A continuación, se listan los requerimientos del producto:

\begin{enumerate}	
	\item Requerimientos funcionales		
	\begin{enumerate}	
		
		\item El robot de exploración ambiental debe poder enviar a la plataforma datos de mediciones de parámetros ambientales, incluyendo los datos de fecha, hora, localización geográfica (que puede ser implementada como un mock inicialmente) y la categorización si es o no un valor crítico.
		\item El robot de exploración ambiental debe incorporar lógica para categorizar los valores medidos de cada parámetro ambiental como valores críticos si:
		\begin{enumerate}				
			\item Representan un máximo o mínimo global sensado hasta el momento.				
			\item Representan un máximo o mínimo local durante el último día.				
		\end{enumerate}			
		\item La plataforma debe poder recibir y almacenar las mediciones de parámetros ambientales enviadas por el robot.
		\item Los datos considerados críticos deben ser almacenados en un sistema inmutable.
		\item La plataforma debe poder procesar las mediciones de parámetros ambientales enviadas por el robot para generar métricas de valor para el usuario de negocio.		
		\item El la plataforma debe brindar dos \textit{front-end} con interfaz web:
			\begin{enumerate}				
				\item El \textit{front-end} para el usuario de negocio.				
				\item El \textit{front-end} para el usuario administrador.				
			\end{enumerate}			
		
		\item El \textit{front-end} para el usuario de negocio debe proveer métricas para visualizar:
			\begin{enumerate}				
				\item Las lecturas históricas almacenadas.				
				\item Agregaciones (máximo, mínimo, promedio, etc) de cada parámetro ambiental agrupado por frecuencias (ventanas de tiempo) y coordenadas geográficas.				
				\item Las referencias a los datos persistidos en blockchain.
			\end{enumerate}			
		\item El \textit{front-end} para el usuario de administración debe permitir:
			\begin{enumerate}				
				\item Acceder a los diferentes recursos utilizados por la herramienta (topics MQTT, \textit{smart contracts}, \textit{buckets}, etc).
				\item Resetear valores y estado.			
			\end{enumerate}			
		\end{enumerate}	

					
	\item Requerimientos no funcionales		
	\begin{enumerate}	
		\item La plataforma debe contar con al menos un \textit{back-end} de procesamiento y acceso a datos operacionales para la lógica de negocio.
		\item La plataforma debe contar con al menos un \textit{back-end} de acceso, procesamiento, almacenamiento de datos analíticos para la generación de métricas.		
		\item El envío de los valores ambientales censados al \textit{back-end} debe ser mediante MQTT.
		\item Las lecturas ambientales categorizadas como críticas deben ser almacenadas en blockchain para garantizar fiabilidad e inmutabilidad.
		\item La gestión de datos almacenados en blockchain debe ser implementada mediante \textit{smart contracts} desplegados en la red.
		\item La interacción con los \textit{smart contracts} debe realizarse desde una dApp.
		\item Los sistemas de transferencia y almacenamiento de datos utilizados deben contar con seguridad, permitiendo encriptación, autenticación y autorización.	
		\end{enumerate}	
		
	\item Requerimientos de documentación		
		\begin{enumerate}			
			\item Video demostrativo.	
			\item Documentación de arquitectura técnica del diseño del sistema.			
			\item Manual de usuario.	
			\item Memoria final.	
		\end{enumerate}	
		

\vspace{4em}
		
		
	\item Requerimiento de testing		
		\begin{enumerate}			
			\item Se debe incluir tests de unitarios de componentes.
			\item Se debe incluir tests funcionales (\textit{smoke test}) del producto general.		
		\end{enumerate}	
	
	\item Requerimientos opcionales		
		\begin{enumerate}			
			\item De infraestructura y despliegue:
				\begin{enumerate}			
					\item Se permite realizar el despliegue de la dApp en un IPFS (preferentemente) o en la nube.					
					\item Se permite agregado hardware al robot para la captura de datos adicionales.
					\item Se permite agregar automatización para la creación de la infraestructura como código.
				\end{enumerate}			
			
			\item De datos:
				\begin{enumerate}			
					\item Se permite almacenar cualquier otro dato adicional sensado o derivado.
					\item Se permite agregar cualquier implementación de gobierno de datos.	
					\item Se permite almacenar cualquier otra métrica o gráfico de explotación de datos adicional.
				\end{enumerate}
		
	\end{enumerate}
\end{enumerate}


\section{7. Historias de usuarios (\textit{Product backlog})}
\label{sec:backlog}

A continuación, se listan las historias de usuario. La ponderación de \textit{story points} se realiza considerando 1 punto = 1 día:

\begin{enumerate}

	\item Arquitectura de la solución - Integración MQTT
	\begin{itemize}
		\item Detalle: como arquitecto, quiero que el envío de los datos desde el robot al \textit{back-end} se realice mediante MQTT para permitir una comunicación asincrónica de baja latencia e integración estándar.
		\item Esfuerzo: 5 puntos
		\item Criterio de aceptación: funcionalidad verificada y documentación.
	\end{itemize}


	\item Arquitectura de la solución - Sistema operacional
	\begin{itemize}
		\item Detalle: como arquitecto, quiero que la arquitectura de tanto los \textit{front-ends} como el \textit{back-end} sean web exponiendo APIs vía servicios REST. La interfaz gráfica debe ser implementada en Angular u otro framework MVC/MVVC de Javascript para permitir un desarrollo dinámico, maximizar la cohesión y reducir el acoplamiento entre componentes.
		\item Esfuerzo: 37 puntos
		\item Criterio de aceptación: funcionalidad verificada y documentación.
	\end{itemize}

	\item Arquitectura de la solución - Sistema analítico
	\begin{itemize}
		\item Detalle: como arquitecto, quiero que las métricas de valor sean generadas de forma \textit{batch} mediante un \textit{pipeline} de datos para poder calcular agregaciones de subconjuntos de datos por rangos seleccionados (por ejemplo, por fechas, por categorías, etc).
		\item Esfuerzo: 27 puntos
		\item Criterio de aceptación: funcionalidad verificada y documentación.
	\end{itemize}

	\item Arquitectura de la solución - Sistema blockchain
	\begin{itemize}
		\item Detalle: como arquitecto blockchain, quiero que la infraestructura seleccionada permita el despliegue de dApps y \textit{smart contracts} de manera escalable, simple y que el desarrollo sea fácil de mantener y para maximizar el tiempo de disponibilidad del producto.
		\item Esfuerzo: 37 puntos
		\item Criterio de aceptación: funcionalidad verificada y documentación.
	\end{itemize}

	\item Desarrollo - Métricas agregadas
	\begin{itemize}
		\item Detalle: como usuario, quiero una métrica que me indique los valores agregados de mínimo, máximo y promedio de cada parámetro ambiental agrupado por fecha y coordenadas para poder brindar una perspectiva de valores normales esperados por parámetro ambiental.
		\item Esfuerzo: 15 puntos
		\item Criterio de aceptación: funcionalidad verificada y documentación.
	\end{itemize}

	\item Desarrollo - Métricas históricas
	\begin{itemize}
		\item Detalle: como usuario, quiero poder ver en series de tiempo los valores medidos para los diferentes parámetros ambientales, incluyendo su media, máximo, mínimo y desviación estándar para poder brindar una perspectiva global de la variación histórica de los parámetros ambientales.
		\item Esfuerzo: 18 puntos
		\item Criterio de aceptación: funcionalidad verificada y documentación.
	\end{itemize}
	
	\item Desarrollo - Gráficos de explotación
	\begin{itemize}
		\item Detalle: como usuario, quiero contar con métricas gráficas (como por ejemplo, \textit{heatmap}, \textit{pie}, etc) para poder visualizar más fácilmente los agregados.
		\item Esfuerzo: 20 puntos
		\item Criterio de aceptación: funcionalidad verificada y documentación.
	\end{itemize}
	
	\item Desarrollo - Smart contracts
	\begin{itemize}
		\item Detalle: como arquitecto, quiero que el acceso a los datos en blockchain se realice por medio de la implementación de \textit{smart contracts} para delegar a la red blockchain la ejecución de lógica de negocio de forma descentralizada y mejorar el aislamiento de componentes.
		\item Esfuerzo: 20 puntos
		\item Criterio de aceptación: funcionalidad verificada, tests y documentación.
	\end{itemize}
	
	\item Desarrollo - dApp
	\begin{itemize}
		\item Detalle: como arquitecto, quiero que el acceso a los \textit{smart contracts} se realice desde una dApp utilizando una biblioteca Web3 para maximizar la integración entre lógica de negocio en blockchain y transferencia de datos hacia y desde la red.
		\item Esfuerzo: 20 puntos
		\item Criterio de aceptación: funcionalidad verificada, tests y documentación.
	\end{itemize}
	
	\vspace{2em}
	\item Infraestructura de desarrollo - Integración continua
	\begin{itemize}
		\item Detalle: como arquitecto, quiero incluir un pipeline de integración continua como parte del proceso de construcción para maximizar la calidad del producto garantizando que por cada commit/push habrá un ciclo de validaciones integral de compilación y testing.
		\item Esfuerzo: 3 puntos
		\item Criterio de aceptación: funcionalidad verificada y documentación.
	\end{itemize}

	\item Seguridad - Encriptación de datos en tránsito
	\begin{itemize}
		\item Detalle: como ingeniero de seguridad, quiero que los datos transmitidos en cada segmento viajen encriptados mediante certificados para garantizar la encriptación punto a punto de datos en tránsito.
		\item Esfuerzo: 2 puntos
		\item Criterio de aceptación: funcionalidad verificada, tests y documentación.
	\end{itemize}

\item Seguridad - Encriptación de datos en reposo
	\begin{itemize}
		\item Detalle: como ingeniero de seguridad, quiero que los datos almacenados estén cifrados para garantizar la encriptación de datos en reposo.
		\item Esfuerzo: 2 puntos
		\item Criterio de aceptación: funcionalidad verificada, tests y documentación.
	\end{itemize}

\item Seguridad - Autorización
	\begin{itemize}
		\item Detalle: como ingeniero de seguridad, quiero que haya asignación y validación de roles para los usuarios que ingresan a los \textit{front-ends }para garantizar el principio de autorización.
		\item Esfuerzo: 2 puntos
		\item Criterio de aceptación: funcionalidad verificada, tests y documentación.
	\end{itemize}

\end{enumerate}


\section{8. Entregables principales del proyecto}
\label{sec:entregables}
Los entregables del proyecto son:
\begin{itemize}
	\item Video demostrativo del uso de la plataforma. 		
	\item Código fuente del firmware.
	\item Código fuente del software.
	\item Documentación:
	\begin{enumerate}				
		\item Manual de usuario.
		\item Memoria final.
		\item Documentación de arquitectura técnica del sistema (hardware y software).
	\end{enumerate}	
	
\end{itemize}


\section{9. Desglose del trabajo en tareas}
\label{sec:wbs}

A continuación, se puede apreciar el conjunto de actividades y tareas que se realizarán durante el desarrollo del proyecto.
\begin{enumerate}

\item Desarrollo cloud (214 h)
	\begin{enumerate}
	\item Análisis y selección de arquitectura cloud (18 h).
	\item Setup de arquitectura cloud (18 h).
	\item POC (\textit{proof-of-concept}) de backend cloud (16 h).
	\item POC front-end negocio (16 h).
	\item POC front-end admin (16 h).
	\item Desarrollo de funcionalidad cloud (39 h).
	\item Testing front-back cloud (5 h).
	\item POC data pipeline analítico (16 h).
	\item Desarrollo del modelo de datos data pipeline analítico (20 h).
	\item Desarrollo de procesamiento de datos para el data pipeline analítico (25 h).
	\item Testing data pipeline analítico (5 h).
	\item POC integración MQTT robot-cloud (5 h).
	\item Integración MQTT robot-cloud (15 h).
	\end{enumerate}

\item Desarrollo blockchain (240 h)
	\begin{enumerate}
	\item Análisis y selección de arquitectura blockchain (34 h).
	\item Setup de arquitectura blockchain (35 h).
	\item POC smart contract (30 h).
	\item Desarrollo smart contracts - persistencia (15 h).
	\item Desarrollo smart contracts - comportamiento (35 h).
	\item POC dApp (35 h).
	\item Desarrollo dApp (40 h).
	\item Testing blockchain (16 h).
	\end{enumerate}

\item Set-up ambiente de integración continua (actividad opcional) (16 h)
	\begin{enumerate}
	\item Set-up imagen Docker con código de proyecto (10 h).
	\item Set-up servicio de integración continua cloud (4 h).
	\item Configuración con Github para tomar los commits y ejecución de builds (2 h).
	\end{enumerate}


\item Documentación (100 h)
	\begin{enumerate}				
	\item Escritura de manual de usuario (24 h).			
	\item Escritura de memoria final (40 h).
	\item Escritura de documentación de arquitectura técnica del sistema (24 h).		
	\item Creación del video demostrativo de uso (12 h).				
	\end{enumerate}	


\vspace{2em}

\item Gestión del proyecto (45 h)
	\begin{enumerate}
	\item Definición de alcance, funcionalidades e historias de usuario (3 h).	
	\item Armado del plan de actividades y tareas (3 h).
	\item Reconocimiento de riesgos (3 h).
	\item Definición de proceso de calidad (2 h).
	\item Confección de documentación de planificación de proyecto (10 h).
	\item Seguimiento y control de hitos, desvíos y riesgos (24 h).	
	\end{enumerate}
		
\end{enumerate}

Cantidad total de horas: (615 h)


\section{10. Diagrama de Activity On Node}
\label{sec:AoN}

A continuación, se detalla la lista de actividades que se realizarán durante el proyecto con los tiempos expresados en días y horas, considerando que habrá solamente un recurso asignado con una dedicación de 3 horas diarias solamente dias hábiles.

\begin{table}[ht]
%\caption{Identificación de los interesados}
%\label{tab:interesados}
\begin{tabularx}{\linewidth}{@{}|l|X|l|l|l|@{}}
\hline
\rowcolor[HTML]{C0C0C0}
Id	& Tarea           						& Duración 	& Dependencia	& Predecesora 	\\ \hline
1	& Set-up integración continua (opcional)	& 5 d / 16 h 		& -        		&  -				\\ \hline
2	& Desarrollo cloud						& 71 d / 214 h		& -				&  -      		\\ \hline
3	& Desarrollo blockchain    				& 80 d / 240 h		& -			 	& -			\\ \hline
4	& Gestión del proyecto 					& 15 d / 45 h		& -				&  - 			\\ \hline
5	& Documentación    						& 33 d / 100 h		& -			 	& -				\\ \hline
\rowcolor[HTML]{C0C0C0}
\multicolumn{2}{|c|}{TOTAL} & \multicolumn{3}{c|}{ 205 días / 615 horas}  \\ \hline
\end{tabularx}
\end{table}

Como se puede apreciar la estimación es de 205 días, siendo en total 41 semanas o 10 meses de trabajo con las restricciones planteadas.

A continuación, se puede apreciar en la figura \ref{fig:activityOnNode}, el camino crítico está formado por la tarea [3] (desarrollo blockchain) y su duración es 80 días.

\begin{center}
 \includegraphics[scale=0.30]{./Figuras/activity-on-node}
 \captionof{figure}{Diagrama de \textit{Activity-On-Node}.}
 \label{fig:activityOnNode}
\end{center}

\section{11. Diagrama de Gantt}
\label{sec:gantt}

A continuación, se puede apreciar el diagrama de Gantt general en la figura \ref{fig:gantt0}, junto con los detalles de sus tareas en las figuras \ref{fig:gantt1}, \ref{fig:gantt2} y \ref{fig:gantt3}.

\begin{center}
 \includegraphics[scale=0.28]{Figuras/gantt-0}
 \captionof{figure}{Diagrama de Gantt general.}
 \label{fig:gantt0}
\end{center}

\begin{center}
 \includegraphics[scale=0.28]{Figuras/gantt-1}
 \captionof{figure}{Diagrama de Gantt detallado (parte 1).}
 \label{fig:gantt1}
\end{center}

\begin{center}
 \includegraphics[scale=0.28]{Figuras/gantt-2}
 \captionof{figure}{Diagrama de Gantt detallado (parte 2).}
 \label{fig:gantt2}
\end{center}

\begin{center}
 \includegraphics[scale=0.28]{Figuras/gantt-3}
 \captionof{figure}{Diagrama de Gantt detallado (parte 3).}
 \label{fig:gantt3}
\end{center}


\section{12. Presupuesto detallado del proyecto}
\label{sec:presupuesto}

El siguiente cuadro presenta los costos en dólares estadounidenses estimados para el proyecto:


\begin{table}[htpb]
\centering
\begin{tabularx}{\linewidth}{@{}|X|c|r|r|@{}}
\hline
\rowcolor[HTML]{C0C0C0}
\multicolumn{4}{|c|}{\cellcolor[HTML]{C0C0C0}COSTOS DIRECTOS} \\ \hline
\rowcolor[HTML]{C0C0C0}
Descripción &
\multicolumn{1}{c|}{\cellcolor[HTML]{C0C0C0}Cantidad} &
\multicolumn{1}{c|}{\cellcolor[HTML]{C0C0C0}Valor unitario} &
\multicolumn{1}{c|}{\cellcolor[HTML]{C0C0C0}Valor total} \\ \hline
Procesamiento cloud &
\multicolumn{1}{c|}{1} &
\multicolumn{1}{c|}{\$ 50,00} &
\multicolumn{1}{c|}{\$ 50,00} \\ \hline
Almacenamiento cloud &
\multicolumn{1}{c|}{1} &
\multicolumn{1}{c|}{\$ 20,00} &
\multicolumn{1}{c|}{\$ 20,00} \\ \hline
Transferencia de datos cloud &
\multicolumn{1}{c|}{1} &
\multicolumn{1}{c|}{\$ 50,00} &
\multicolumn{1}{c|}{\$ 50,00} \\ \hline
Alojamiento IPFS &
\multicolumn{1}{c|}{1} &
\multicolumn{1}{c|}{\$ 20,00} &
\multicolumn{1}{c|}{\$ 20,00} \\ \hline
Horas de ingeniería &
\multicolumn{1}{c|}{615} &
\multicolumn{1}{c|}{\$ 5,00} &
\multicolumn{1}{c|}{\$ 3.075,00} \\ \hline
Varios &
\multicolumn{1}{c|}{1} &
\multicolumn{1}{c|}{\$ 20,00} &
\multicolumn{1}{c|}{\$ 20,00} \\ \hline
Costos indirectos &
\multicolumn{1}{c|}{1} &
\multicolumn{1}{c|}{\$ 485,00} &
\multicolumn{1}{c|}{\$ 485,00} \\ \hline
\rowcolor[HTML]{C0C0C0}
\multicolumn{3}{|c|}{TOTAL} &
\multicolumn{1}{c|}{\$ 3.720,00} \\ \hline
\end{tabularx}%
\end{table}

Tomando la cotización Peso/Dólar del Banco de la Nación Argentina vigente a la fecha 16 de agosto de 2024 equivalente a \$ 900 por dólar, el monto total en pesos argentinos es \$ 3.348.000,00.


\section{13. Gestión de riesgos}
\label{sec:riesgos}

\begin{enumerate}
\item Riesgo de demora
\begin{itemize}
	\item Severidad (S): 9 - Teniendo en cuenta que solo habrá un recurso (el alumno) asignado al proyecto desarrollando el producto, una demora en cualquier tarea puede implicar demora en la fecha de entrega.
	\item Ocurrencia (O): 5 - Dado el desafío de innovación planteado por el alcance él la probabilidad de demora es media.
\end{itemize}

\vspace{3em}

\item Riesgo de no contar con toda la funcionalidad deseada
\begin{itemize}
	\item Severidad (S): 9 - El no cumplimiento con la funcionalidad deseada pone en riesgo el éxito del proyecto.
	\item Ocurrencia (O): 5 - Dado el desafío de innovación planteado por el alcance, la probabilidad de no contar con toda la funcionalidad es media.
\end{itemize}

\item Riesgo de calidad insuficiente
\begin{itemize}
	\item Severidad (S): 4 - La calidad insuficiente no pone en riesgo el cumplimiento con la funcionalidad pero si compromete la estabilidad y resistencia a fallas del producto, por lo que puede desencadenar en un producto poco o menos confiable.
	\item Ocurrencia (O): 4 - Se estima que con las técnicas empleadas durante el desarrollo del producto, este riesgo tiene una baja probabilidad de ocurrencia.
\end{itemize}


\item Riesgo de desvío en costos
\begin{itemize}
	\item Severidad (S): 5 - La ocurrencia de este riesgo impacta en los costos del proyecto, pero no imposibilita ni demora la entrega del producto.
	\item Ocurrencia (O): 8 - Teniendo en cuenta que los precios son estimados sin haber cerrado la definición de la arquitectura final y la inflación en Argentina, es muy probable que exista un desvío en costos.
\end{itemize}

\item Riesgo de indisponibilidad de recursos
\begin{itemize}
	\item Severidad (S): 5 - Al momento de realizar el presente plan se identifican ciertos recursos y se asume que será posible disponer de ellos. No obstante, existe el riesgo de que esto no suceda así, lo que puede dificultar por ejemplo, adquirir \textit{tokens} de prueba por medio de \textit{Faucets} para el acceso a redes de desarrollo blockchain; realizar el despliegue de la dApp; acceder a servicios cloud que soporten todas las prestaciones necesarias por la arquitectura; o haya recursos no directamente asociados al proyecto pero cuya ausencia lo afectan, como ser fallas en el acceso a Internet, el mal funcionamiento de la computadora utilizada para su desarrollo, fallas en el hardware del robot que hagan necesario su reparación, etc.
	\item Ocurrencia (O): 2 - Se espera que la probabilidad de ocurrencia de este riesgo sea realmente baja.
\end{itemize}

\end{enumerate}

b) Tabla de gestión de riesgos:      (El RPN se calcula como RPN=SxO)

\begin{table}[htpb]
\centering
\begin{tabularx}{\linewidth}{@{}|X|c|c|c|c|c|c|@{}}
\hline
\rowcolor[HTML]{C0C0C0}
Riesgo 													& S & O & RPN & S* & O* & RPN* \\ \hline
Riesgo de demora en la entrega							& 9 & 5 & 45 &	9  &  1  & 9    \\ \hline
Riesgo de no contar con toda la funcionalidad deseada		& 9 & 5 & 45 & 	10  & 1 &  10    \\ \hline
Riesgo de calidad insuficiente							& 4 & 4 & 16 &  	4 &  2 &   8  \\ \hline
Riesgo de desvío en costos								& 5 & 8 & 40 & 	5  & 3  &  15   \\ \hline
Riesgo de indisponibilidad de recursos					& 5 & 2 & 10 & 	-  & -  &   -   \\ \hline
\end{tabularx}%
\end{table}

Criterio adoptado: se tomarán medidas de mitigación en los riesgos cuyos números de RPN sean mayores a 15.

Nota: los valores marcados con (*) en la tabla corresponden luego de haber aplicado la mitigación.

\vspace{2em}

c) Plan de mitigación de los riesgos que originalmente excedían el RPN máximo establecido:
\begin{enumerate}
	\item Riesgo de demora en la entrega: las posibles causas del evento asociado están vinculadas a situaciones no controlables ni predecibles que impactan de alguna manera en la disponibilidad de tiempo o alguno de los recursos necesarios para la realización del proyecto. Con el fin de cumplir con la entrega de la funcionalidad en la fecha acordada se consideran como posibles acciones de mitigación la eliminación (o no realización) de otras tareas dentro del plan, como por ejemplo, tareas de documentación, de testing, y de ser necesario, de desarrollo de funcionalidad. Se asume que la eliminación de estas tareas ponen en riesgo la calidad del producto y/o contar con toda la funcionalidad esperada.
	\begin{itemize}
		\item Nueva Severidad (S*): 9 - No cambia.
		\item Nueva Ocurrencia (O*): 1.
	\end{itemize}
	
	\item Riesgo de no contar con toda la funcionalidad deseada: las posibles causas del evento asociado están vinculadas a situaciones que impactan de alguna manera en la viabilidad o desarrollo de alguna de las funcionalidades en el tiempo planificado. Por este motivo se considera como herramientas de mitigación: sacrificar algún otro entregable (como por ejemplo la cobertura de testing y/o documentación, lo cual puede implicar sacrificar calidad, mantenibilidad y/o usabilidad respectivamente), o bien redefinir el alcance y funcionalidad en base a lo que es posible desarrollar.
	\begin{itemize}
		\item Nueva Severidad (S*): 10 - Dado que tras la mitigación se incrementa el impacto por pérdida de calidad.
		\item Nueva Ocurrencia (O*): 1 - Se reduce mucho la probabilidad de ocurrencia dado que se agrega tiempo para el desarrollo de funcionalidad eliminando el tiempo empeñado para el desarrollo de tests y/o documentación.
	\end{itemize}
	
	
	\item Riesgo de calidad insuficiente: las posibles causas del evento asociado están vinculadas a la falta de estabilidad del producto, sea por una arquitectura demasiado compleja o ineficiente, o la presencia de bugs en el desarrollo. Para mitigar este problema se plantea emplear más tiempo en el desarrollo de pruebas de concepto y definición de la arquitectura, además incrementar las prácticas de testing y CI/CD (como actividad opcional) siempre que esto no dispare el riesgo 1, que podría generar una demora una demora en el proyecto.
	\begin{itemize}
		\item Nueva Severidad (S*): 4 - No cambia.
		\item Nueva Ocurrencia (O*): 2 - Se reduce la probabilidad de que esto suceda.
	\end{itemize}	
	
	\item Riesgo de desvío en costos: las posibles causas del evento que dispara este riesgo están vinculadas a la dificultad de estimar los costos de procesamiento, almacenamiento y transferencia de datos tanto en la nube pública como en la red blockchain a utilizar, debido a que la arquitectura no ha sido definida al momento de la escritura del presente plan y los costos vinculados a su implementación varían dependiendo del proveedor, red y tecnologías utilizadas. Además, indirectamente asociado a esto, es posible olvidar estimar algún componente. Finalmente el impacto de la inflación en Argentina puede ser otro causal del desvío en costos. Para mitigar el primer factor se agrega el item Varios / Imprevistos a la tabla de materiales con el fin de proveer holgura en el caso de no contemplar algún componente adicional. Para mitigar el factor inflación se plantean los precios en dólares americanos.
	\begin{itemize}
		\item Nueva Severidad (S*): 5 - Esto no varía.
		\item Nueva Ocurrencia (O*): 3 - Disminuye la probabilidad de ocurrencia dado que la inflación del dólar estadounidense es menor que la del peso argentino.
	\end{itemize}
\end{enumerate}




\section{14. Gestión de la calidad}
\label{sec:calidad}

\begin{enumerate}
		\item Funcionalidad del \textit{front-end} de negocio: métricas y gráficos de explotación.
		\begin{enumerate}				
			\item Verificación previa a la entrega: se verificará esta funcionalidad mediante la ejecución de tests funcionales y unitarios.	
			\item Validación: el cliente validará de forma manual la ejecución de la funcionalidad de los reportes y gráficos de explotación en el producto final.			
		\end{enumerate}		
		
		\item Funcionalidad del \textit{front-end} de administración: control de la plataforma.
		\begin{enumerate}				
			\item Verificación previa a la entrega: se verificará esta funcionalidad mediante la ejecución de tests funcionales y unitarios.	
			\item Validación: el cliente validará de forma manual la ejecución de la funcionalidad de administración de la plataforma en el producto final.			
		\end{enumerate}		
	
		\item Funcionalidad de la DApp, almacenamiento en blockchain y ejecución de \textit{smart contracts}.
		\begin{enumerate}				
			\item Verificación previa a la entrega: se verificará esta funcionalidad mediante la ejecución de tests funcionales y unitarios.	
			\item Validación: el cliente validará de forma manual la ejecución de la funcionalidad en la dApp y de los \textit{smart contracts} accediendo a la red blockchain mediante el navegador web.			
		\end{enumerate}			
	
		\item Documentación técnica, manual de usuario y memoria final.
		\begin{enumerate}				
			\item Verificación previo a la entrega: se verificará mediante la revisión de los documentos.			
			\item Validación: el cliente validará la completitud y claridad de los documentos entregados.			
		\end{enumerate}			
		
		\item Testing.
		\begin{enumerate}				
			\item Verificación previo a la entrega: Se verificará el nivel de cumplimiento, y de ser posible la cobertura, de los casos de test por funcionalidad en el prototipo final.
			\item Validación: el cliente validará el reporte de los tests de integración.
		\end{enumerate}			
		
		
\end{enumerate}



\section{15. Procesos de cierre} 
\label{sec:cierre}

\begin{itemize}
	\item Pautas de trabajo que se seguirán para analizar si se respetó el Plan de Proyecto original:
	 - Responsable: \authorname:
	\begin{itemize}			
		\item Se evaluarán los requerimientos y los objetivos alcanzados frente a los planteados en el plan.
		\item Se pondrá especial interés en verificar si se cumplieron los objetivos de tiempo y funcionalidad propuestos.
	\end{itemize}	    	
	
	\item Identificación de las técnicas y procedimientos útiles e inútiles que se emplearon, y los problemas que surgieron y cómo se solucionaron:
	 - Responsable: \authorname:
	\begin{itemize}			
		\item Se evaluará cuál fue la configuración que mejores resultados arrojó para los objetivos planteados en el plan.
		\item Se identificarán nuevas herramientas o procedimientos, en caso que corresponda.
	\end{itemize}	    	
	
	\item Indicar quién organizará el acto de agradecimiento a todos los interesados, y en especial al equipo de trabajo y colaboradores - Responsable: \authorname :
	\begin{itemize}			
		\item Luego de la presentación del proyecto mediante la defensa pública, se procederá a agradecer a todas las personas que participaron del desarrollo del proyecto, al director y a las autoridades de la CEIoT.
	\end{itemize}	
\end{itemize}

\end{document}



